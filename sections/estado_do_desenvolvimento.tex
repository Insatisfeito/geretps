\section{Estado do desenvolvimento}
No sentido de colocar em prática todo o planeamento do desenvolvimento da aplicação, seguiu-se de forma minimamente próxima o plano de trabalho já demonstrado na forma de diagramas de Gantt com pequenas alterações de prioridade no que diz respeito à implementação de algumas fases.

Nesse sentido após a construção da base da aplicação nas tecnologias que seriam usadas procedeu-se à prototipagem em HTML de algumas páginas.

Neste momento é possível encontrar já em fase quase final a página principal da aplicação que sofreu algumas iterações até se aproximar do pretendido.

Passado esta primeira fase de prototipagem começou-se por garantir a persistência de dados através da implementação do modelo de dados concebido durante a análise de requisitos bem como proceder à sua população com dados de teste.

No que diz respeito aos diferentes utilizadores das aplicação e à sua autenticação no sistema foram implementadas funcionalidades de registo e entrada  no sistema de forma a garantir uma gestão fiel das funcionalidades que cada tipo de utilizador tem acesso, bem como garantir coerência de dados para cada utilizador individual.

Após garantir a autenticação no sistema aos seus utilizadores,bem como o registo específico para cada um dos tipos de utilizadores (docentes e alunos), iniciou-se aimplementação das funcionalidades básicas do sistema, nomeadamente o \textit{CRUD (create, read, updateand delete)} da aplicação, ou seja, a criação, leitura, actualização e remoção de todas as entidades representadas no sistema (como por exemplo turnos, projetos, grupos, avaliações, etc).

Após uma implementação básica dessas funcionalidades decidiu-se iterar sobre as funcionalidades ligadas ao tipo de utilizador aluno e refazê-las de raiz de forma a ir de encontro ao que realmente são as funcionalidades que pretendemos disponibilizar aos alunos da aplicação.

Nesse sentido iniciou-se o desenvolvimento do painel do aluno que se torna o  centro de utilização da aplicação por parte deste tipo de utilizador.
Aqui é possível consultar os projetos em desenvolvimento por parte do utilizador, assim como os projetos que se vão iniciar e os que já foram terminados.
Também é possível consultar as últimas notificações do sistema assim como se tornam acessíveis todas as unidades curriculares em que cada aluno se encontra inscrito.

De seguida prosseguiu-se ao aprofundamento da implementação das funcionalidades sobre projetos do lado do aluno, tendo sido desenvolvida uma página onde é possível consultar as informações disponibilizadas para o projeto, bem como as fases, as suas notificações, o grupo de trabalho e as últimas entregas.
A partir daqui é também possível consultar páginas de entrega e fases de um projeto onde são disponibilizadas informações como os ficheiros obrigatórios, a pauta de avaliações, as entregas feitas e onde é possível efectuar novas entregas.
Para além disso também se tornam acessíveis na página de um projeto páginas que permitem a gestão de grupos e consulta de enunciados.

Ainda do lado do aluno e no que diz respeito às unidades curriculares em que estão inscritos é disponibilizada um página que permite a procura de unidades curriculares e a sua inscrição mediante a aceitação por parte do docente do respetivo pedido, bem como a funcionalidade de consulta das unidades curriculares em que está inscrito e as que estão com a inscrição pendente, assim com a opção para anular uma inscrição.
Assim, é também tornada possível a consulta de todas as avaliações das diferentes unidades curriculares em que um aluno se encontra inscrito.

De referir que todas as funcionalidades que já foram implementadas já se encontram numa fase de maturação considerável e que estão bastante próximas da sua versão final apesar de estarem ainda sujeitas a pequenas alterações, tanto no sentido do seu funcionamento como no sentido da própria interface, que por sua vez já se encontra preparada para o acesso a partir de várioss tipos de dispositivos independentemente da resolução.

Em jeito de resumo já é possível fornecer aos alunos utilizadores do sistema uma aplicação que lhes permite gerir as suas unidades curriculares, ingressar em projetos, fazer entregas, consultar pautas e avalições, criar os seus grupos de trabalho, entre outras funcionalidades, garantido assim grande parte das funcionalidades planeadas para este tipo de utilizador.

De seguida virar-se-ão as atenções para as funcionalidades no plano de utilização do sistema por parte dos docentes onde por agora se encontram implementadas funcionalidades básicas de \textit{CRUD}.


\newpage
