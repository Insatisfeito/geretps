\section{Conclusão}

Chegando ao final da primeira fase deste projeto é possível destacar algumas conclusões e decisões
tomadas desde o início até ao ponto atual.

Relativamente ao levantamento de funcionalidades e análise de requisitos conseguiu-se retratar
corretamente o sistema em análise, cobrindo todo o espetro de funcionalidades necessárias para
o bom funcionamento da solução que se irá desenvolver para este problema de gestão de trabalhos 
práticos académicos em concreto.
Foram detetadas as principais necessidades dos diversos tipos de utilizadores 
para o qual este sistema é idealizado e também foi possível proceder a um 
planeamento bem definido no que diz respeito ao futuro do desenvolvimento deste 
projeto.

No que diz respeito às decisões tecnológicas foi decidido proceder à 
implementação da aplicação web usando a linguagem Ruby on Rails, não só pela 
experiência que os elementos do grupo já possuem com a respetiva 
\textit{framework}, mas também pelas suas características de prototipagem rápida 
e pelas filosofias de desenvolvimento que a regem.
Para além disso, a utilização desta framework irá ser bastante positiva no ponto 
em que se seguirá uma abordagem Agile no desenvolvimento de todo este sistema.

Por fim, a fase de desenvolvimento já foi iniciada e já foram criados vários prótotipos 
relativos às interfaces disponibilizadas para a interação entre os utilizadores 
e o sistema. Atualmente os resultados parecem ser positivos, pois a análise ao 
resultado destes protótipos relevam que estes se destacam pela simplicidade de 
intuitividade bastante elevada, o que é o objetivo idealizado para este 
sistema.

\newpage
