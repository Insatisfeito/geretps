\subsection{Open Archives Initiative Protocol for Metadata Harvesting} % (fold)
\label{sub:open_archives_initiative_protocol_for_metadata_harvesting}

Uma das funcionalidades do \emph{GereTPs} é a exportação de dados recorrendo ao protocolo OAI-PMH. Para tal implementou-se um \emph{Servidor de Dados (Data Provider)}.

Os pedidos ao servidor são feitos via http através de parâmetros \emph{GET} e a resposta para cada pedido é dada sob a forma de \emph{XML}.

Os pedidos suportados pelo servidor e respetivos argumentos são:

\begin{description}
	\item[GetRecord] Informação detalhada sobre um \emph{Record}. No caso do \emph{GereTPs} cada \emph{Record} corresponde a uma entrega de um projeto. Os argumentos são:
	\begin{itemize}
		\item identifier
		\item metadataPrefix
	\end{itemize}

	\item[Identify] Informação sobre o repositório

	\item[ListRecords] Lista de Records que podem ser filtrados por categoria e/ou por data. Os argumentos são:
	\begin{itemize}
	  	\item from
	  	\item until
	  	\item resumptionToken
	  	\item metadataPrefix
	\end{itemize}

	\item[ListIdentifiers] É uma abreviação do \emph{ListRecords}, em que disponibiliza um \emph{header} para cada \emph{Record}. Os argumentos são:
	\begin{itemize}
	  	\item from
	  	\item until
	  	\item resumptionToken
	  	\item metadataPrefix
	\end{itemize}

	\item[ListMetadataFormats] Lista todos os formatos de representação de um \emph{Record}

	\item[ListSet] Lista de categorias. No caso do \emph{GereTPs} as categorias são as Instituições, Cursos e Disciplinas, sendo um Curso uma subcategoria de Instituição e Disciplina uma subcategoria de Curso
\end{description}

Nos verbos em que é retornado uma lista de elementos (\emph{ListSet, ListRecords, ListIdentifiers}), a resposta obtida é paginada, isto é sempre que um utilizador faz um pedido é retornado uma lista com o máximo de 100 elementos. Caso haja mais resultados, é retornado no \emph{XML} um \emph{resumptionToken}, que quando usado num pedido retorna os resto dos elementos, respeitando a regra dos 100 elementos por resposta.

\begin{verbatim}
	//PEDIDO
	localhost:3000/oai?verb=ListSets

	//Resposta
	<OAI-PMH xmlns="http://www.openarchives.org/OAI/2.0/" xmlns:xsi="http://www.w3.org/2001/XMLSchema-instance" xsi:schemaLocation="http://www.openarchives.org/OAI/2.0/ http://www.openarchives.org/OAI/2.0/OAI-PMH.xsd">
	<responseDate>2014-07-14T12:10:02Z</responseDate>
	<request value="ListSets">http://localhost:3000/oai</request>
	<ListSets>
	<set>
	<setSpec>Universidade%20do%20Porto</setSpec>
	<setName>Universidade do Porto</setName>
	</set>
	<set>
	<setSpec>Universidade%20de%20Coimbra</setSpec>
	<setName>Universidade de Coimbra</setName>
	</set>
	<set>
	<setSpec>Universidade%20T%C3%A9cnica%20de%20Lisboa</setSpec>
	<setName>Universidade Técnica de Lisboa</setName>
	</set>
	<set>
	<setSpec>Universidade%20de%20Lisboa</setSpec>
	<setName>Universidade de Lisboa</setName>
	</set>
	<set>
	<setSpec>Universidade%20do%20Minho</setSpec>
	<setName>Universidade do Minho</setName>
	</set>
	<set>
	<setSpec>
	Universidade%20do%20Minho:Mestrado%20em%20Engenharia%20Inform%C3%A1tica
	</setSpec>
	...
	<setName>Instituto Superior de Paços de Brandão</setName>
	</set>
	<resumptionToken completeListSize="121" cursor="0">74d04e736360b180118b35db88461416</resumptionToken>
	</ListSets>
	</OAI-PMH>
\end{verbatim}
% subsection open_archives_initiative_protocol_for_metadata_harvesting (end)
