\subsection{Utilitário de linha de comandos}

O principal objetivo do sistema é facilitar ao máximo a gestão de alunos, trabalhos práticos e avaliações por parte do utilizador. \\

Para facilitar a realização de algumas tarefas e melhorar o desempenho dos utilizadores avançados do sistema, foi desenvolvido um utilitário de linha de comandos que permite realizar várias ações importantes do sistema de forma mais rápida e eficiente.

Segue-se o menu de ajuda do utilitário onde estão descritas as funcionalidades implementadas na versão atual:

\begin{verbatim}
USAGE: geretps <command> [<args>]

Commands:
  init                                Create an empty geretps folder or 
                                        reinitialize an existing one.
  login -E <email> -P <password>      Get authentication token and store 
                                        authentication data.
  list <entity>                       List all objects of type entity 
                                        belonging to the user.
  show <entity> <id>                  Show the object of type entity with 
                                        the specified identifier.
  download <entity>                   Download all objects of type entity 
                                        belonging to the user.
  help                                Show this help message and quit.

Description:
  ...

Examples:
  geretps list projects
  geretps show project 1

See 'geretps <command> help' for more information on a specific command.
\end{verbatim}

Seguem-se alguns exemplos de utilização:

\begin{verbatim}
> geretps list projects

#3    Ficha de avaliação Nº2
#8    Ficha de avaliação Nº1
#9    Ficha de avaliação Nº3
#2    myAcademia - Gestor de Currículos Académicos
#1    Projecto Integrado - GereTPs
#7    Projecto ASP.NET
#6    Projecto C + JAVA - Transitários LEI
#5    Projecto C - Arranjo peças retangulares
#4    Projeto HASKELL - Parser JOIN
\end{verbatim}

\begin{verbatim}
> geretps show project 1

IDENTIFIER:         1
NAME:               Projecto Integrado - GereTPs
DESCRIPTION:        O que se pretende neste projeto...
BEGIN DATE:         2013-10-13T00:00:00.000Z
END DATE:           2014-07-14T00:00:00.000Z
MIN ELEMS:          2
MAX ELEMS:          4
\end{verbatim}

Um dos comandos mais importantes, desenvolvido na versão atual do utilitário, é o comando \verb%download%.\\

Este comando permite fazer download das entidades especificadas, no caso das entregas, o utilizador pode assim sincronizar o seu diretório atual com todas as entregas efetuadas em todos os projetos a que pertence, e assim não necessita de aceder ao sistema e fazer download manualmente das entregas. Essa tarefa fica assim muito simplificada com a utilização do utilitário desenvolvido.\\

Segue-se um exemplo da estrutura de diretorias criada pelo comando.

\begin{verbatim}
   Projecto_C_-_Arranjo_pecas_retangulares
      INFO.xml
   Projecto_Integrado_-_GereTPs
      Fase_1
         Grupo_1
            DELIVERY#1-2014-05-12T02:23:57_274Z
               G1_Diagrams.zip
               G1_Presentation.pdf
               G1_Report.pdf
               G1_Source.zip
               INFO.xml
            INFO.xml
         Grupo_2
            DELIVERY#3-2014-05-12T02:23:57_285Z
               INFO.xml
            INFO.xml
         Grupo_3
            DELIVERY#4-2014-05-12T02:23:57_360Z
               INFO.xml
            INFO.xml
         Grupo_4
            DELIVERY#5-2014-05-12T02:23:57_364Z
               INFO.xml
            INFO.xml
         Grupo_5
            DELIVERY#6-2014-05-12T02:23:57_369Z
               INFO.xml
            INFO.xml
         Grupo_6
            DELIVERY#7-2014-05-12T02:23:57_373Z
               INFO.xml
            INFO.xml
         INFO.xml
      Fase_2
         INFO.xml
      Fase_3
         INFO.xml
      Fase_4
         INFO.xml
      INFO.xml
   Projeto_HASKELL_-_Parser_JOIN
      Fase_1
         Grupo_1
            DELIVERY#2-2014-05-12T02:23:57_281Z
               INFO.xml
            INFO.xml
         INFO.xml
      INFO.xml
\end{verbatim}

De referir que o referido comando, em todas as diretorias criadas, cria um ficheiro \verb%INFO.xml% com todas as informações relativas ao contexto atual.
