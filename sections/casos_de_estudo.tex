\section{Casos de estudo}
De forma a verificar a viabilidade da aplicação decidiu-se usar três casos reais. Nesse sentido irá se explorar cada um dos casos de forma a explorar como a aplicação resolve os problemas expostos.

\subsection{Engenharia de Linguagens}
\label{sub:engenharia_de_linguagens}

No primeiro caso tem-se o Projeto Integrado de Engenharia de Linguagens do Mestrado em Engenharia Informática da Universidade do Minho.
Engenharia de Linguagens tem um turno e quatro docentes.

Quanto ao Projeto Integrado, este tem que ser feito por grupos com o máximo de três elementos. Existem quatro fases de entrega (as 3 primeiras tem uma nota qualitativa e a última fase vale 20 valores). Relativamente à entrega, em cada fase é obrigatório entregar um relatório e um conjunto de slides usados na apresentação. Na última fase além do relatório, torna-se obrigatório enviar o código desenvolvido assim como outros ficheiros que o grupo ache relevante.

O processo começa com o registo do docente responsável, caso o docente já possua uma conta tem de efetuar \emph{login} na aplicação. Após o \emph{login} é necessário criar a cadeira de Engenharia de Linguagens, durante este processo será necessário preencher campos relativos ao nome da instituição e do curso. Caso já  exista o curso ou a instituição, a disciplina será associada aos campos existentes. No fim de criar a disciplina, o docente será encaminhado para o painel da disciplina criada. Dentro do painel da disciplina, o docente responsável pode adicionar docentes à disciplina.

Procede-se para a criação do projeto. Ao criar um novo projeto o docente adiciona o enunciado, adiciona as várias fases do projeto, indica que os grupos só podem ter três elementos. Em cada fase indica que tem que ser enviado um relatório e os \emph{slides} da apresentação em formato \emph{PDF}. Por fim tem a opção de lançar o projeto ou se apenas o quer deixar visível para o resto da equipa docente, para o caso de haverem alterações. Após o projeto estar criado o docente é encaminhado para o painel do projeto.
Um aluno uma vez registado e autenticado inscreve-se na cadeira e acede ao painel do projeto. No painel do projeto cria o seu grupo e faz uma submissão. É então reencaminhado para um formulário onde escreve um resumo do trabalho feito e envia os ficheiros necessários, concluindo assim a submissão.

Voltando ao docente, este acede novamente ao painel do projeto e abre a página de entrega de um projeto submetido. Após analisar o relatório ou qualquer outro ficheiro enviado, carrega em avaliar e atribui a nota ao grupo ou a cada elemento individualmente, podendo também adicionar comentários sobre a nota atribuída.

\subsection{Estatística}
\label{sub:estat_stica}

No segundo caso tem-se como exemplo um projeto da cadeira de Estatística do curso de Economia. Esta cadeira tem quatro turnos e dois docentes. Quanto ao projeto tem que ser feito em grupos de dois ou três elementos, restringido a alunos do mesmo turno, tem apenas uma fase, existe um ficheiro com dados para auxilio da realização do projeto e no momento da entrega além do relatório tem-se que enviar os resultados obtidos no \emph{SPSS} que estão no formato \emph{SAV}.

O processo começa pelo registo da cadeira no sistema após o registo e identificação do docente. Assumindo que todos os alunos já se inscreveram na disciplina é feito a inscrição dos alunos em cada turno por parte do docente. De seguida cria-se o projeto no qual se indica o enunciado do projeto, o ficheiro de dados, restringe-se os grupos para que estes sejam constituídos por elementos do mesmo turno e que a dimensão deste seja de dois a três elementos e indica-se quais os ficheiros obrigatórios no momento da entrega. Por último os alunos submetem o projeto feito e o docente indica a nota do mesmo tal como acontece no primeiro caso.

\subsection{Desafios de Proramação da Universidade do Minho}
\label{sub:desafios_de_prorama_o_da_universidade_do_minho}

No terceiro e último caso pretende-se submeter os desafios algorítmicos dos Desafios de Programação da Universidade do Minho (DPUM). Estes desafios são feitos individualmente e são corregidos automaticamente pela aplicação. Não é necessário entregar relatório e não existe data limite de submissão.

Quanto aos Desafios de Programação da Universidade é dirigido por um único docente e não existem turnos. Na submissão de um desafio é obrigatório que haja uma \emph{Makefile} e que o nome do executável deverá ser igual o ao nome indicado pelo docente.

Assumindo que já existe o registo e autenticação do docente no sistema, procede-se para o registo do DPUM no sistema, o campo referente ao curso é deixado em branco.

Na criação de um desafio indica-se que se pretende que haja correção automática, faz-se \emph{upload} dos ficheiros de \emph{input} e \emph{output} e que os grupos sejam de um único elemento. Para o aluno no momento da submissão não é necessário criar grupo. Após a submissão de um desafio é enviado um \emph{mail} para o aluno com os resultados da correção automática.

\newpage
