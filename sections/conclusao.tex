\section{Conclusão}
Chegado o final da segunda fase deste projeto é possível destacar algumas conclusões e decisões que tiveram destaque durante esta etapa de desenvolvimento.

Uma das decisões tomadas, passou por inicialmente garantir a implementação das funcionalidades básicas de todo o sistema, tornando assim posssível manipular todos os dados que circulam internamente, possibilitando a inserção, leitura, atualização e remoção dos mesmos. Após isto, decidiu-se proceder a um refinamento dessas funcionalidades numa ótica direcionada à utilização da aplicação por parte do aluno, refazendo-as de raíz de forma a garantir um funcionamento o mais fiel possível ao que foi planeado e que colmate o melhor possível as necessidades dos alunos. Pretende-se na próxima etapa criar um foco no desenvolvimento das funcionalidades relacionadas diretamente com os docentes e com a comunidade em geral (no que diz respeito à pesquisa e acesso aos projetos já desenvolvidos).

De notar também as funcionalidades de autenticação e registo, onde se distinguem os registos de docentes e alunos de forma tornar mais intuitvo o uso do sistema, fornecendo funcionalidades específicas e experiências de utilização desenhadas e idealizadas propositadamente para cada tipo de utilizador.

No que diz respeito às conclusões, deve-se destacar o que provaram ter sido boas escolhas na primeira fase do projeto por parte do grupo, nomeadamente as tecnologias a usar e as metodologias de trabalho, bem como interação interna do grupo. Para além disso também foi possível construir uma página principal que transparece bem os objetivos da aplicação, tanto em termos visuais como de conteúdo. O mesmo pode ser dito das interfaces já implementadas, nomeadamente relativas ao painel do aluno que parecem ser uma boa resposta à necessidade dos alunos utlizadores do sistema. 

Por fim, também é digno de nota o resultado final desta etapa que agrada o grupo e transparece o empenho e trabalho que tem sido dedicado.


\newpage

