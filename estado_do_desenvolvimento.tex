\section{Estado do desenvolvimento}

Depois de toda a fase de planeamento, iniciou-se o desenvolvimento do projeto.

Pretende-se que o desenvolvimento seja um processo incremental, começando por definir as funcionalidades principais primeiro e só depois evoluir para as restantes.

Relativamente à base de dados, optou-se por definir a base de dados completa logo à partida, isto porque as nossas funcionalidades principais exigem alterações num grande número de tabelas com um elevado número de dependências.

Para efetuarem-se alterações na base de dados, em \textbf{Ruby on Rails}, existem as \textbf{Migrações}. Migrações são \textit{scripts} escritas em \textbf{Ruby} que utilizam API disponibilizada pelo Rails para criação, alteração e remoção de tabelas da base de dados. À semelhança de outras ações do \textbf{ActiveRecord}, as migrações também são independentes do motor de base de dados utilizado. 

Nesta fase, já foram definidas todas as migrações para criação das tabelas enunciadas na secção do \textbf{Repositório de Informação}.

Outra fase já ultrapassada do desenvolvimento foi a definição dos \textbf{Modelos}.
O componente de Modelo do Rails é um conjunto de classes que usam o ActiveRecord, uma classe ORM que mapeia objetos em tabelas da base de dados. O ActiveRecord usa convenções de nome para determinar os mapeamentos, utilizando uma série de regras convencionadas que devem ser seguidas para que a configuração seja a mínima possível.

Os Modelos do nosso sistema são os enunciados na subsecção \textbf{Diagrama de Classes}, da secção \textbf{Modelo de Dados}.

Dentro dos modelos já estão definidas todas as relações entre as diferentes entidades do sistema, assim como algumas restrições dos dados.

Para além da definição das Migrações e dos Modelos avançamos na construção em HTML das páginas principais.
