\section{Análise do problema}
No meio académico, principalmente em áreas mais focadas numa vertente prática, muitas vezes
os alunos são postos à prova através de trabalhos práticos. Neste caso, após a avaliação, os trabalhos
práticos são arquivados pelos docentes e esquecidos permanentemente numa prateleira, até um dia
serem deitados fora.

No sentido de promover a investigação e a partilha de conhecimento, torna-se não só necessário
criar uma plataforma de gestão de disciplinas universitárias e os seus trabalhos práticos, como também
tornar possível a partilha de todo o conhecimento que é inevitavelmente inerente aos resultados
do desenvolvimento destes.

No momento as ferramentas que existem no mercado, se por um lado permitem toda a gestão de
unidades curriculares de uma universidade, pecam por se focarem por um funcionamento fechado
para dentro da instituição e consequentemente fechado para dentro dos cursos e das unidades curriculares.
Nesse sentido o sistema retratado neste relatório, preenche a lacuna existente entre o que é desenvolvido
num âmbito académico e o resto da população.

Como resultado deste tipo de ponte entre os dois mundos, espera-se conseguir manter uma plataforma
que não só incentive a investigação tanto dentro do mundo académico, como profissional e individual,
como também incentive a partilha de conhecimento e fomente o desenvolvimento das
áreas de atuação condizentes com cada trabalho prático disponível na
plataforma.

Para além disso, torna-se imperativo criar um modelo genérico de publicação de
trabalhos práticos para dessa forma se tornar muito mais acessível o consumo de
informação para os utilizadores da plataforma.

No entanto também se torna necessário o foco no lado mais académico e das
necessidades dos docentes e alunos. Neste sistema todo o processo de
interatividade entre os utilizadores e a aplicação, permite tanto aos docentes
como aos alunos realizarem a gestão das suas unidades curriculares e trabalhos
práticos criando assim um sistema que cumpre as necessidades académicas internas dos docentes
e alunos bem como estabelece um ponto de ligação entre a comunidade académica e
o trabalho desenvolvido nesse meio, e o resto das pessoas com interesse nas
áreas abrangidas pelos projetos disponíveis.
