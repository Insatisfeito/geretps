\begin{abstract}

  Este relatório cobre todo o processo de planeamento, desenvolvimento e
  documentação de um sistema que resolva um problema abordado na UCE de
  Engenharia de Linguagens na forma de Projeto Integrado, nomeadamente
  a gestão de trabalhos práticos académicos desde a sua criação pelos docentes, à resolução pelos alunos
  e à publicação dos mesmos ao resto da comunidade \textit{online}.

  A abordagem na resolução deste projeto passa por aliar todos os conhecimentos
  adquiridos nos módulos integrantes de Engenharia de Linguagens, passando por
 Engenharia Gramatical, \textit{Scripting} no Processamento de Linguagem Natural, Processamento
 Estruturado de Documentos e Análise e Transformação de Software.

 Neste sentido pretende-se aliar ao desenvolvimento deste sistema conhecimentos
 como gramáticas de atributos, ambientes de desenvolvimento estruturais e
 orientados à semântica, representação de manipulação de conhecimento com
 eficiência, automatização de tarefas e transformações, utilização de expressões
 regulares, linguagens DSL, corpora, automatização de testes para diferentes
 linguagens de programação, manipulação de documentos estruturados, utilização
 de XML, XSL e XSL-FO, documentos anotados, publicação de conhecimento na web,
 entre outros.

\end{abstract}
